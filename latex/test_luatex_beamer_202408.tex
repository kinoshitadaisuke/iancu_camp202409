\documentclass[t]{beamer}

% Time-stamp: <2024/08/01 19:56:41 (UT+8) daisuke>

%
% preamble
%

\title{\bf Making a presentation slide using Lua\TeX}
\author{Kinoshita Daisuke}
\date{August 2024}

%
% main body of document
%
\begin{document}

% title page
\begin{frame}
 \titlepage
\end{frame}

% new page
\begin{frame}{Making a presentation slide using Lua\TeX}
 \begin{itemize}
  \item Presentation slides can be produced easily by Beamer.
  \item To learn about Beamer, visit following web page.
	\begin{itemize}
	 \item Beamer: https://ctan.org/pkg/beamer
	\end{itemize}
  \item A new slide page can be created by using {\tt frame}
	environment.
  \item Mathematical formulae can be used.
  \item The absolute magnitude of an object $M$ can be calculated by
	\[
	M = m - 5 \log_{10} d + 5,
	\]
	where $m$ is apparent magnitude and $d$ is distance in parsec.
  \item An example of a table.
	\begin{center}
	 \begin{tabular}{|l|r|r|r|l|}
	  \hline
	  Star & App. Mag. & Abs. Mag. & Dist. & Type \\
	  \hline \hline
	  Sun & $-26.74$ & $+4.83$ & 1 au & G2V \\
	  Sirius & $-1.46$ & $+1.43$ & 2.67 pc & A0V \\
	  Vega & $+0.03$ & $+0.58$ & 7.68 pc & A0V \\
	  Betelgeuse & $+0.50$ & $-5.85$ & 168 pc & M1I \\
	  \hline
	 \end{tabular}
	\end{center}
 \end{itemize}
\end{frame}

\end{document}